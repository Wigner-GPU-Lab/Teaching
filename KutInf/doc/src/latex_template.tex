\documentclass[a4paper, 12pt]{article}
\usepackage{graphicx}

\usepackage[utf8]{inputenc}
\usepackage[magyar]{babel}

% a szép matematikai szimbólumokért
\usepackage{amssymb}
\usepackage{amsmath}

% ha táblázatban szeretnénk egyesített sorokat is
\usepackage{multirow}

% horizontal line
\usepackage{hhline}

% eps formátumú ábrák --> pdflatex fordításhoz!!
\usepackage{epsfig}

% egyenletekhez, pl mátrixok írására
\usepackage{array}

% ha betűszíneket is szeretnénk használni
\usepackage{color}

% Margók egyéni beállításai
\usepackage{anysize}
\marginsize{1.64cm}{1.64cm}{1.2cm}{2.4cm} %\left right top bottom

% vakszöveg
\usepackage{lipsum}
\usepackage{blindtext}

% A HIVATKOZÁSOKHOZ HASZNÁLT CSOMAGOK. RÉSZLETESEBBEN LD: google -> latex bibtex
\usepackage[numbers, square, comma, sort&compress]{natbib}
%\usepackage[format=hang,labelsep=period]{caption}

\usepackage[unicode]{hyperref}   % ezzel a hivatkozások linkké válnak
\usepackage{bookmark}
\hypersetup{bookmarksopen={true}}
\hypersetup{bookmarksopenlevel={2}}
\hypersetup{bookmarksnumbered={true}}
\hypersetup{
  colorlinks,%
  citecolor=red,%
  filecolor=black,%                                                                                                                                               
  linkcolor=blue,%
  urlcolor=green
}
\numberwithin{equation}{section}          % ezekkel tudod beállítani, hogy milyen felbontásig menjen a hivatkozás
\numberwithin{figure}{subsection}
%\numberwithin{table}{section}          % ha kikommenteled, akkor csak simán számozva lesz.

%%%%%%%%%%%%%%%% Néhány dolog a fancy kinézethez

\frenchspacing
\setlength{\parskip}{2ex}
\setlength{\headsep}{0,4cm}
\setlength{\headheight}{4pt}

% fej- es lábléc
\usepackage{fancyhdr}
\usepackage{fancyref}
\usepackage{fancyvrb}
\pagestyle{fancy}

\renewcommand{\headrulewidth}{0,05pt}
\renewcommand{\footrulewidth}{0pt}


\fancyhf{}
\fancyhead[RE]{{ \nouppercase{\leftmark}} }
\fancyhead[LO]{{ \nouppercase{\leftmark}} }
\cfoot{--~\thepage~--}


%%%%%%%%%%%%%%%%%%%%%%%%%%%


%%%%%%%%%%%%%%%%%%%%%%%%%%%%%%%%%%%%%%%%%%%%%%%%%%%%%%%%%%%%%%%%%%%%%%%%%%%%

\begin{document}

% Címoldalt lehet egyszerűen a \maketitle paranccsal is. Ha kissé részletesebb
% címre van szükség, azt lehet így is, kézzel megadva mindent.
\begin{titlepage}   
\begin{center}
\thispagestyle{empty}  

\vspace*{0.7cm}
\rule{\linewidth}{0.5mm} \\[3mm]
\vspace*{0.7cm}

{\LARGE Számítógépes alapismeretek}

\vspace*{0.7cm}
\rule{\linewidth}{0.5mm} \\[3mm]
\rule{\linewidth}{0.5mm} \\[3mm]



{\Large Gyakorlat\\}

\vspace*{0.7cm}
\rule{\linewidth}{0.5mm} \\[3mm]
  {\small \LaTeX\ példafájl} \\[3mm]
  \vspace*{1cm}
{\footnotesize Írta: Bíró Gábor} \\
{\tiny biro.gabor@wigner.mta.hu}

  \vspace*{2cm}

\begin{figure}[h!]
\begin{center}
\includegraphics[width=0.5\textwidth]{./elte.eps}
\end{center}
\end{figure}

\end{center}
\end{titlepage}

\newpage

%%%%%%%%%%%%%%%%%%%%%%%%%%%%%%%%%%%%%%%%%%%%%%%%%%%%%%%%%%%%%%%%%%%%%%%%

\thispagestyle{empty}  

\begin{abstract}
  Kivonat-bevonat. Ez a példa tartalmaz minden olyat, amire a jövőben szükség lehet - de 
  közel sem mindent, amire a \LaTeX képes. A következő bekezdés pedig egy vakszöveg, (,,töltelékszöveg'', 
  \textit{placeholder}), amivel a dokumentum megjelenését lehet demonstrálni.

  Előbb azonban vegyük észre, hogy az első bekezdésben a \textit{töltelékszöveg} szó kilóg 
  az oldal szélén. Előfordul, hogy a \LaTeX\ nem tudja tökéletesen elválasztani a szavakat.
  Hasonlóan, a \LaTeX\ parancs után kell egy szóköz, amit kényszerítve kell berakni.
  Ezek nem szép hibák, erre figyelni kell. Helyesen így néz ki ugyanaz a bekezdés: 

  Kivonat-bevonat. Ez a példa tartalmaz minden olyat, amire a jövőben szükség lehet - de 
  közel sem mindent, amire a \LaTeX\ képes. A következő bekezdés pedig egy vakszöveg, (,,töl\-te\-lék\-szö\-veg'', 
  \textit{placeholder}), amivel a dokumentum megjelenését lehet demonstrálni.

  \lipsum[1]

\end{abstract}

% ---------------------------- T A R T A L O M J E G Y Z É K ----------------------------
\newpage \vspace*{2cm}
\thispagestyle{plain}                                                                                                                                             
\pagenumbering{roman} \setcounter{page}{1}
\tableofcontents

\newpage \vspace*{2cm}
\thispagestyle{plain}
\listoffigures

\newpage \vspace*{2cm}
\thispagestyle{plain}
\listoftables

\newpage \vspace*{2cm}

\pagenumbering{arabic} \setcounter{page}{1}
\pagestyle{fancy}

% ---------------------------- M A G A   A   D O L G O Z A T ----------------------------

%%%%%%%%%%%%%%%%%%%%%%%%%%%%%%%%%%%%%%%%%%%%%%%%%%%%%%%%%%%%%%%%%%%

\section{Bevezetés}
\label{sec:bev}
Itt egy hangzatos bevezetés szerepel.

Ez egy következő bekezdés. Lehet használni \textbf{félkövér}, \textit{dőlt} és 
\underline{aláhúzott} betűket is. Az "idézőjelet" a magyar nyelvben pedig helyesen ,,így'' 
kell használni.

%%%%%%%%%%%%%%%%%%%%%%%%%%%%%%%%%%%%%%%%%%%%%%%%%%%%%%%%%%%%%%%%%%%%%%%%%%%%%%%%%%%%%%%
\section{Egy fejezet}
\label{sec:fejezet}

Ez egy fejezet kezdete.

\subsection{Egy alfejezet}
\label{subsec:alfejezet}

Ez meg egy alfejezet.

\subsubsection{Alalfejezet}
\label{subsubsec:alalfejezet}

Ez pedig egy alalfejezet. Nevezetesen a \ref{sec:bev}. bevezetés után kezdődő \ref{sec:fejezet}. fejezet \ref{subsec:alfejezet}. alfejezetének \ref{subsubsec:alalfejezet}. alalfejezete, amiben meghivatkozom a \ref{fig:abra}. ábrát és a \ref{tab:tab}. táblázatot (helyesebben: \aref{tab:tab}. táblázatot). Meg a \cite{artic:elso} és \cite{artic:masodik, artic:harmadik, artic:negyedik} hivatkozásokat is. Ahhoz, hogy minden hivatkozás a helyére kerüljön, a \ref{eq:eq1}. egyenlet után következő lépéseket kell követned. Ha hiba van egy hivatkozásban, akkor az így fog megjelenni: \cite{ezegynemletezohivatkozas}, illetve \ref{esezmegmilehet}. A \texttt{\textbackslash cite}\footnote{Ez példa a lábjegyzetre, melyben leírom, hogy hogyan lehet a \textbackslash\ karaktert megjeleníteni. Mivel a \textbackslash\ karakterrel a \LaTeX-ben parancsot kell adni, parancsba kell adni, hogy most ezt a speciális karaktert is meg szeretnénk jeleníteni, így: \texttt{\textbackslash textbackslash}. Ez a forráskódban pedig így nézett ki: \texttt{\textbackslash texttt\{\textbackslash textbackslash textbackslash\}}. A forráskódba belenézve azt hiszem lehet érezni, hogy ezt kiírni már elég hosszú lenne.} paranccsal lehet irodalomjegyzékben definiált referenciára hivatkozni. Erre is több mód van; ehhez a dokumentumhoz a \textbf{bibtex} csomagot használtam. Ez nem a mostani anyag része, de későbbi használatra innen meg lehet tanulni.

Ez egy új bekezdés. Az előző fejezet a forráskódban egyetlen sor volt, ami \textsc{vim}-ben 
szerkesztve kényelmetlen lehet. Éppen ezért érdemes figyelni a forráskód tördelésére is.

\begin{equation}
a+b=c
  \label{eq:eq1}
\end{equation}

Ez volt az egyenlet. 

\subsubsection{A \LaTeX\ dokumentum lefordítása}
A fordításhoz végrehajtandó lépések:

\begin{itemize}
  \item[a)] van \textit{eps}-től eltérő képformátum (\textit{png, jpg} vagy más) a dokumentumban:
  \begin{enumerate}
    \item pdflatex fajlnev.tex
    \item bibtex fajlnev.aux
    \item pdflatex fajlnev.tex
    \item pdflatex fajlnev.tex
  \end{enumerate}
  \item[b)] nincs \textit{eps}-től eltérő képformátum a dokumentumban:
  \begin{enumerate}
    \item latex fajlnev.tex
    \item bibtex fajlnev.tex
    \item latex fajlnev.tex
    \item latex fajlnev.tex
    \item dvipdf fajlnev.tex
  \end{enumerate}
  \item harmadik opció nincs. Ezzel csak demonstrálom, hogy mi az \texttt{itemize} alapértelmezett szimbóluma.
\end{itemize}

Adott esetben azért lehet szükség az a) pontra, mert a \texttt{latex} parancs csak az \emph{eps} 
kiterjesztésű képeket eszi meg. \textbf{\textcolor{red}{Amikor csak lehet, igyekezzünk ragazkodni 
ehhez a formátumhoz!!}} 

Amennyiben más formátumú képet is muszáj használnunk (pl. jelen dokumentumban), úgy fordításnál a 
\texttt{pdflatex} parancsot kell használni\footnote{Megjegyzés: modern operációs rendszerekben valójában már a 
\texttt{latex} is \texttt{pdflatex}, csak \texttt{dvi} módban futtatva.}. Ekkor viszont szükség lehet az 
\emph{epsfig} package-re is az \textit{eps} képekhez.

Végül, a lusták figyelmébe ajánlom a \texttt{lazy.sh} fájlt, amely pontosan ezeket a sorokat tartalmazza 
(kiegészítve a fordítás során keletkező ideiglenes fájlok eltakarításával). Ennek ha futtatási jogot adunk a 

\texttt{chmod +x lazy.sh}

paranccsal\footnote{Ismételjük át a linux parancsokat is!}, majd a \texttt{./lazy.sh} paranccsal futtatjuk, 
magától lefuttatja a fordítás fáradtságos műveletsorát.

\paragraph{Megjegyzés}
Gyakorlaton többeknél előfordult, hogy a \texttt{latex latex.tex} kétszeri, hiba nélküli lefuttatása után a 
\texttt{dvipdf latex.dvi} egy hasonló hibaüzenettel tért vissza: 

\noindent % ezzel megakadájozzuk, hogy a következő bekezdés valóban beljebb kezdődjön
user@machine:$\sim$\$/latex/dvipdf latex.tex \\ % <-- új sor. 
% Vegyük észre, hogy a $ jelet \$ paranccsal hoztam elő, míg a ~ jelet a \sim parancs adja, amit $ jelek közé,
% tehát matematikai módba kell rakni!
page 1 may be too complex to print

Némi google után a legjobb tippem, hogy ez még egy ,,régi'' korból ittmaradt \textbf{feature}, amikor még az 
eszközöknek meglehetősen véges volt a rendelkezésre álló memóriája. Ha a dokumentumunkban sok nagyfelbontású 
grafikát szeretnénk haszálni (ilyeneknek számít pl a szöveg is!), akkor előfordulhatott, hogy ez a memória 
nem is volt elég. Ezt az információt egyébként egy konfigurációs fájlból szedte ki a program, így ha ezt a fájlt 
módosítja az ember, a probléma megkerülhető. Valószínűleg azért nem jelentkezett a probléma mindenkinél 
(például nálam sem), mert az újabb \LaTeX\ verzióknál már default több a konfigban elérhető memória.

\begin{figure}[h!]
\begin{center}
\includegraphics[width=0.9\textwidth]{./comic.jpg}
\end{center}
  \caption{\textit{Ez az ábrafelirat.} Az ábra jól szemlélteti az egyes programozási 
  nyelvek sajátosságait.}
\label{fig:abra}
\end{figure}

\paragraph{Ez egy paragrafus}
\lipsum[2-4]
\begin{equation}
  \varepsilon=\sum_{i=1}^{n-1} \frac{1}{\Delta x}\int\limits_{x_i}^{x_{i+1}}\left\{\frac{1}{\Delta x}
  \left[(x_{i+1}-x)y_i^*+(x-x_i)y_{i+1}^*\right]-f(x)\right\}^2dx
\end{equation}

Végül pedig egy táblázat következik.

\begin{table}[h!]
\caption{Ez meg egy példatáblázat.}
\label{tab:tab}
\begin{center}
\small
  \begin{tabular}{r@{,}lc||r}

\hline
\hline
    \multicolumn{2}{c}{Első oszlop} & Harmadik oszlop & \multirow{2}{*}{Negyedik oszlop} \\
     \multicolumn{2}{c}{}           & (De vajon hova lett a második..?) & \\
\hline
\hline
  1 & 2 & 3 & 4 \\
  5 & 6 & 7 & 8 \\
\hline
  9 & 10 & 11 & 12 \\
\hline
\hline

\end{tabular}
\end{center}
\end{table}


%%%%%%%%%%%%%%%%%%%%%%%%%%%%%%%%%%%%%%%%%%%%%%%%%%%%%%%%%%%%%%%%%%%%%%%%%%%%%%%%%%%%%%%%%%

% ez ugyanazt tudja, mint a \newpage, azzal a különbséggel, hogy hogyha vannak még függő 
% ábrák vagy táblázatok, amik még nem lettek elhelyezve, akkor az most itt megtörténik, és 
% csak utána megyünk tovább. A \newpage most nekünk pl hatástalan lenne.
\clearpage 

\vspace*{2cm}

% Mivel *-al jelöljük meg az új section-t, nem fog bekerülni a tartalomjegyzékbe, hacsak kézzel hozzá nem adjuk
\addcontentsline{toc}{section}{Köszönetnyilvánítás}
\section*{Köszönetnyilvánítás}
Ez fontos rész, még új oldalt is kezdtünk hozzá. Mindenképp köszönd meg témavezetődnek, családodnak, a Mikulásnak, meg akinek még gondolod. ;)

%%%%%%%%%%%%%%%%%%%%%%%%%%%%%%%%%%%%%%%%%%%%%%%%%%%%%%%%%%%%%%%%%%%%%%%%%%%%%%%%%%%%%%%%%%5

\newpage \vspace*{2cm}

\addcontentsline{toc}{section}{Irodalomjegyzék}
\bibliographystyle{abeld}       
% van egy csomó féle/fajta stílus, google segít bennük. Én ezt használtam a diplomamunkámhoz. 
% A helyes működéshez az kell, hogy az abeld.bst ugyanebben a mappában legyen.
\bibliography{references}       
% a references.bib tartalmazza egyenként a hivatkozásokat, így a tex fájlod tisztább maradhat.
%
% 
%%%%%%%%%%%%%%%%%%%%%%%%%%%%%%%%%%%%%%%%%%%%%%%%%%%%%%%%%%%%%%%%%%%%%%%%%%%%%%%%%%%%%%%%

% ----------------------------------- A P P E N D I X -----------------------------------

\newpage \vspace*{2cm}

\pagestyle{fancy}

\appendix
\section{Ez egy függelék}
\label{app:fugg}

És igen, ezeket is lehet \ref{app:fugg}. módon hivatkozni.

% -------------------------------- N Y I L A T K O Z A T --------------------------------

% Nagyon hosszú dokumentumoknál (pl könyv) érdemes több részre bontani a forráskódot is.
% Az \input parancs után következő fájl egyszerűen "bemásolódik", így a fordító egyben 
% fogja látni az egészet.
\input nyilatkozat.tex  % EZT NE HAGYD KI! OTT KELL LENNIE A DOLGOZAT VÉGÉN!


\end{document}

Ez pedig már nem fog megjelenni, mivel az end document után vagyunk.
